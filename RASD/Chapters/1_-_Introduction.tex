In the real world, the process of finding suitable candidates for internships is often a complex and time-consuming task for companies, while university students face significant challenges in identifying opportunities that align with their skills, interests, and career aspirations. Companies have to consider numerous applications, many of which may not meet their requirements, and produce internship descriptions that attract the right talent. On the other hand, students are frequently left to navigate partial information sources, leading to inefficiencies and missed opportunities.
This disparity between the supply and demand for internships is caused by the lack of "ad hoc" tools to support efficient matchmaking. The success of an internship is often guaranteed by factors such as the relevance of candidates' skills to the offered projects, the clarity of internship descriptions, and the availability of resources like mentorship and training. Without a straight process, companies may struggle to identify candidates who fit their needs, and students may find it difficult to show their potential.
These challenges highlight the need for a unified platform that bridges the gap between students and companies, providing a mutually beneficial ecosystem where both parties can connect, evaluate, and collaborate. S\&C comes as a solution to address these pain points, offering a structured approach to simplifying and optimizing the internship process for all stakeholders involved.
\newpage

\section{Purpose}
\label{sec:purpose}%

\subsection{Goals}
\label{subsec:goals}%
\newcounter{g}
\setcounter{g}{1}
\newcommand{\cg}{\theg\stepcounter{g}}

Below there's a table that lists all the goals of the \verb|S&C| system:
\begin{center}
    \renewcommand{\arraystretch}{2}
    \begin{longtable}{ l p{0.8\linewidth} } 
        \hline
        \textbf{ID} & \textbf{Description}                                                                   \\
        \hline
        %%todo, mettere insieme 1-3 e 1-2
        G\cg  & Allows Companies to advertise their internship offers to find the most suitable students \\ \hline
        G\cg  & Allows Students to look for internships based on their needs \\ \hline
        G\cg  & Allows Students to be recommended to companies using keyword-based and statistical matching \\ \hline
        %%G\cg  & Allows Students to be notified when a new internship offer is published \\ \hline
        G\cg  & Supports selection process by helping manage interviews and also finalize the selections \\ \hline
        G\cg  & Provides suggestions to companies regarding how to make their offers more appealing for students \\ \hline
        G\cg  & Provides suggestions to students how to make their CVs more appealing for companies \\ \hline
        G\cg  & Allows stakeholders to monitor the progress of internships, report issues, and track outcomes \\ \hline
        G\cg  & Allows Universities to monitor the situation of ongoing internships and interrupt them when necessary \\ \hline
        \caption{The goals.}
        \label{tab:goals_tab}%
    \end{longtable}
\end{center}

\newpage


\section{Scope}
\label{sec:scope}%
The S\&C platform serves as a comprehensive system for matching students with internships and supporting related workflows. It acts as a hub for collaboration and interaction among three primary user groups: students, companies, and universities. Students are the primary user group, using the platform to create detailed profiles, upload and maintain updated CVs, search for internships manually, or receive personalized recommendations that better suit to their academic background, skills, and preferences. They can also apply to internships, participate in interviews with the help of the platform and finally provide feedback on their experiences, thereby contributing to the platform's constant improvement.
Companies utilize S\&C to manage their internship offerings and streamline recruitment. They can advertise internships with comprehensive descriptions, specifying required skills and qualifications, and benefit from a recommendation system that identifies suitable candidates. Companies can also review applications, shortlist applicants and schedule interviews within the platform. For the post-selection, they can provide feedback on students' performance, helping refine the matching algorithms and contributing to system insights.
Universities play a crucial supervisory role, ensuring the integrity and success of internships. They monitor ongoing internships, address complaints raised by students or companies and analyze trends to enhance their internship programs. Additionally, universities use the platform to ensure compliance with educational and legal standards, providing a secure and supportive environment for all parties involved.
Through the integration of these interactions, the S\&C platform simplifies the internship process while creating a cooperative environment that mutually supports students, companies, and universities.
\subsection{World phenomena}
\label{subsec:world_phenomena}%
\newcounter{wp}
\setcounter{wp}{1}
\newcommand{\cwp}{\thewp\stepcounter{wp}}
\begin{center}
    \renewcommand{\arraystretch}{2}
    \begin{longtable}{ l p{0.8\linewidth} } 
        \hline
        \textbf{ID} & \textbf{Description}                                                \\
        \hline
        WP\cwp      & A student wants to find an internship that meets his or her needs and goals\\
        \hline
        WP\cwp      & A student creates or update their CVs, reflecting their real-world skills and experiences   \\
        \hline
        WP\cwp      & A company decides to offer a new internship, defining its internship requirements and benefits \\
        \hline
        WP\cwp      & A student decides to accept or reject recommendations based on their preferences.          \\
        \hline
        WP\cwp      & A company decides to accept or reject recommendations based on their preferences                 \\
        \hline
        WP\cwp      & A company conducts interviews and evaluate candidates directly                   \\
        \hline
        WP\cwp      & Internships proceed in the real world, with students working on projects as described by the companies                                            \\
        \hline
        WP\cwp      & Stakeholders (students, companies, universities) identify issues during internships and report them to the appropriate parties.      \\
        \hline
        WP\cwp      & A university decides to interrupt an internship, based on the student or company complaints                       \\
        \hline
        \caption{World Phenomenas.}
        \label{tab:worldph_tab}%
    \end{longtable}
\end{center}

\subsection{Shared phenomena}
\label{subsec:shared_phenomena}%
\newcounter{sp}
\setcounter{sp}{1}
\newcommand{\csp} {\thesp\stepcounter{sp}}
\begin{center}
\renewcommand{\arraystretch}{2}

\paragraph{Shared Phenomena - World Controlled}

\begin{longtable}{ l p{0.8\linewidth} }
    \hline
    \textbf{ID} & \textbf{Description} \\ 
    \hline
    SP\csp & A student signs up to the system or logs in if already registered, using the academic email address \\ 
    \hline
    SP\csp & A company signs up to the system or logs in if already registered \\ 
    \hline
    SP\csp & A student uploads his CV \\ 
    \hline
    SP\csp & A student uploads additional personal information to guarantee better matching \\ 
    \hline
    SP\csp & A student chooses to accept a recommended internship \\ 
    \hline
    SP\csp & A student provides feedback and suggestions on how to improve a recently completed internship \\ 
    \hline
    SP\csp & A student provides problems or complaints about an ongoing internship \\ 
    \hline
    SP\csp & A company provides additional information to guarantee better matching \\ 
    \hline
    SP\csp & A company publishes and manages internship offers through the system interface \\ 
    \hline
    SP\csp & A company chooses to accept a recommended student for a specific internship \\ 
    \hline
    \caption{Shared Phenomena - World Controlled.}
    \label{tab:sharedph_world_tab}%
\end{longtable}

\paragraph{Shared Phenomena - Machine Controlled}

\begin{longtable}{ l p{0.8\linewidth} }
    \hline
    \textbf{ID} & \textbf{Description} \\ 
    \hline
    SP\csp & The system provides personalized recommendations for internships to a student \\ 
    \hline
    SP\csp & The system provides personalized recommendations for candidates to a company \\ 
    \hline
    SP\csp & The system supports companies by organizing interviews and collecting structured questionnaire responses \\ 
    \hline
    SP\csp & The system offers personalized suggestions for improving CV to a student \\ 
    \hline
    SP\csp & The system offers personalized suggestions for improving job postings to a company \\ 
    \hline 
    SP\csp & The system sends a notification to a student of a new internship offer that meets his needs \\ 
    \hline 
    SP\csp & The system enables users to submit and manage complaints, facilitating resolution with relevant stakeholders. \\ 
    \hline
    \caption{Shared Phenomena - Machine Controlled.}
    \label{tab:sharedph_machine_tab}%
\end{longtable}

\end{center}

\section{Definition, Acronyms, Abbreviations}
\label{sec:definition_acronyms_abbreviations}%
\begin{table}[H]
    \begin{center}
        \begin{tabular}{ |l|l| }
            \hline
            \textbf{Acronyms} & \textbf{Definition}                              \\
            \hline
            RASD & Requirements Analysis and Specification Document \\ \hline
            S\&C & Students\&Companies \\ \hline
            CV & Curriculum Vitae \\ \hline
        \end{tabular}
        \caption{Acronyms used in the document.}
        \label{tab:acronyms}%
    \end{center}
\end{table}


\section{Revision history}
\label{sec:revision_history}%
 \begin{itemize}
     \item Version 1.0 - 22/12/2024
 \end{itemize}

\section{Reference Documents}
\label{sec:reference_documents}%
\begin{itemize}
    \item Assignment document
    \item CreatingRASD (lecture slides)
\end{itemize}


\section{Document Structure}
\label{sec:document_structure}%
\begin{itemize}
    \item \textbf{Introduction}: A general introduction to the goals, the phenomena and the scope of
            the system. It provides general but exhaustive information about what the RASD
            document is going to explain.
    \item \textbf{Overall Description}: A general description of the product to be,its requirements and the scenarios that might occur.
    \item \textbf{Specific Requirements:} All software requirements are explained using scenarios,
            use-case diagrams and activity diagrams. It focuses on the specific requirements and provides a more
            detailed analysis of external interface requirements, functional requirements and performance ones.
    \item \textbf{Formal Analysis Using Alloy:} This section includes Alloy code that describes the
            model and shows its soundness and correctness.
    \item \textbf{Effort Spent:} Effort spent by all team members shown as the list of all the activities
            done during the realization of this document
    \item \textbf{References:} References to documents that this project was developed upon.
\end{itemize}
