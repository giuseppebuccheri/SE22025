\section{Purpose}
\label{sec:purpose}%
In the real world, the process of finding suitable candidates for internships is often a complex and time-consuming task for companies, while university students face significant challenges in identifying opportunities that align with their skills, interests, and career aspirations. Companies must consider numerous applications, many of which may not meet their requirements, and produce internship descriptions that attract the right talent. However, students are often left to navigate partial information sources, leading to inefficiencies and missed opportunities.
This disparity between the supply and demand for internships is caused by the lack of "ad hoc" tools to support efficient matchmaking. The success of an internship is often guaranteed by factors such as the relevance of candidates' skills to the offered projects, the clarity of internship descriptions, and the availability of resources such as mentorship and training. Without a straight process, companies may struggle to identify candidates who fit their needs, and students may find it difficult to show their potential.
These challenges highlight the need for a unified platform that bridges the gap between students and companies, providing a mutually beneficial ecosystem where both parties can connect, evaluate, and collaborate. S\&C comes as a solution to address these pain points, offering a structured approach to simplifying and optimizing the internship process for all involved stakeholders.

\section{Scope}
\label{sec:scope}%
The S\&C platform serves as a comprehensive system for matching students with internships and supporting related workflows. It acts as a hub for collaboration and interaction among three primary user groups: students, companies, and universities. Students are the primary user group, using the platform to create detailed profiles, upload and maintain updated CVs, search for internships manually, or receive personalized recommendations that better suit their academic background, skills, and preferences. They can also apply to internships, participate in interviews with the help of the platform, and finally provide feedback on their experiences, thereby contributing to the platform's constant improvement.
Companies utilize S\&C to manage their internship offerings and streamline recruitment. They can advertise internships with comprehensive descriptions, specifying required skills and qualifications, and benefit from a recommendation system that identifies suitable candidates. Companies can also review applications, shortlist applicants and schedule interviews within the platform. For post-selection, they can provide feedback on students' performance, helping refine the matching algorithms, and contributing to system insights.
Universities play a crucial supervisory role, ensuring the integrity and success of internships. They monitor ongoing internships, address complaints raised by students or companies, and analyze trends to enhance their internship programs. In addition, universities use the platform to ensure compliance with educational and legal standards, providing a secure and supportive environment for all parties involved.
Through the integration of these interactions, the S\&C platform simplifies the internship process while creating a cooperative environment that mutually supports students, companies, and universities.

\section{Definition, Acronyms, Abbreviations}
\label{sec:definition_acronyms_abbreviations}%
\begin{table}[H]
    \begin{center}
        \begin{tabular}{ |l|l| }
            \hline
            \textbf{Acronyms} & \textbf{Definition}                              \\
            \hline
            DD & Design Document \\ \hline
            S\&C & Students\&Companies \\ \hline
            CV & Curriculum Vitae \\ \hline
            DBMS & Data Base Management System \\ \hline
            API & Application Programming Interface\\ \hline
            REST & Representational State Transfer \\ \hline
        \end{tabular}
        \caption{Acronyms used in the document.}
        \label{tab:acronyms}%
    \end{center}
\end{table}

\section{Revision history}
\label{sec:revision_history}%
 \begin{itemize}
     \item Version 1.0 - 07/01/2025
 \end{itemize}

\section{Reference Documents}
\label{sec:reference_documents}%
\begin{itemize}
    \item Specification Document : "Assignment RDD AY 2024-2025"
    \item "CreatingDD" (lecture slides)
\end{itemize}


\section{Document Structure}
\label{sec:document_structure}%
\begin{itemize}
    \item \textbf{Introduction}: The first chapter provides an overview of the purpose, scope, and objectives of the Design Document. It also introduces the definitions, acronyms and abbreviations used throughout the document and explains the structure.
    \item \textbf{Architectural Design}: This chapter describes the overall system architecture, including component diagrams, deployment views, run-time views, and the interfaces between components. It points out how the system architecture supports the requirements.
    \item \textbf{User Interface Design}: This chapter focuses on the design of user interfaces. Includes an overview of the main UI elements, descriptions of each interface, and how they interact with the system to support user experience.
    \item \textbf{Requirements Traceability}: This section maps the system's requirements to the design elements, ensuring that all requirements are covered and showing how the design fulfills them.
    \item \textbf{Implementation, Integration and Test Plan}: This chapter presents the implementation strategy, the integration plan, and the testing approach. Details how the system components will be developed, combined and verified to ensure a functional and robust product.
    \item \textbf{Effort Spent}: A summary of the contributions and time spent by each team member during the development of the Design Document.
    \item \textbf{References}: A list of references to all documents, articles, and tools that informed the creation of this Design Document.
\end{itemize}

